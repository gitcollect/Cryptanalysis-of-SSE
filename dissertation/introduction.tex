\chapter*{Introduction}

À l'heure du tout numérique, le volume de nos données ne cesse de croître au fil
des jours, nous conduisant parfois à les externaliser et à les confier à des
opérateurs \textit{Cloud}. Certaines de ces données, comme les documents
juridiques, bancaires et médicaux, les brevets industriels ou plus simplement
nos e-mails, peuvent être de nature sensible ou bien confidentielle nous
obligeant à faire confiance à l'opérateur \textit{Cloud}.\\

Afin de s'affranchir de la confiance accordée à l'opérateur \textit{Cloud}, il
est alors important de chiffrer ces données. La solution idéale à cette
problématique repose sur le chiffrement dit homomorphe permettant d'effectuer
tout type d'opérations sur des données chiffrées \cite{ref7}. Malheureusement,
elle reste actuellement de l'ordre de la recherche académique.\\

Une solution possible serait de chiffrer préalablement les données à
externaliser. Par définition, elles seront inintelligibles par l'opérateur
\textit{Cloud} et donc inexploitables. Paradoxalement, si ces données chiffrées
sont inintelligibles par l'opérateur \textit{Cloud}, elles le sont aussi pour
nous. Nous serons alors dans l'impossibilité de manipuler nos propres données.
Bien sûr, nous pourrions penser qu'il nous suffirait de télécharger toutes ces
données chiffrées puis de les déchiffrer localement. Nous pourrions ainsi
réaliser les opérations voulues mais cela va à l'encontre des avantages attendus
d'un opérateur \textit{Cloud}. En effet, un des points forts d'un opérateur
\textit{Cloud} est de nous permettre de stocker une grande quantité de données
en ligne et donc accessible de n'importe quel endroit. Cette solution est alors
loin d'être efficace et même parfois impraticable si le volume de données
sauvegardées est très important.\\

La solution actuelle est de déposer ces données en clair sur l'opérateur
\textit{Cloud}. Cela nous permet de les exploiter sans aucune contrainte. Il
est alors légitime de s'inquiéter sur la confidentialité et la sécurité de ces
données transmises. En effet, nous ne possédons aucun contrôle sur l'opérateur
\textit{Cloud}, ce dernier peut à tout moment consulter les données stockées,
les dupliquer ou encore les transmettre à des tiers. De plus, nous n'avons
aucune garantie que l'infrastructure de l'opérateur soit suffisamment sécurisée
et qu'elle puisse ainsi empêcher toute intrusion malicieuse.\\

Une autre possibilité intéressante est de se restreindre à la fonctionnalité de
recherche sur données chiffrées. En effet, une telle solution nous permet
d'avoir nos données en sécurité sur l'opérateur \textit{Cloud} tout en ayant la
possiblité de récupérer les données chiffrées que nous cherchons spécifiquement.
Cette solution est réalisable grâce aux schémas de recherches sur données
chiffrées, dits SSE pour \textit{Searchable Symmetric Encryption}. En effet,
ces schémas, même s'ils ne permettent pas toutes les opérations que peut offrir
le chiffrement homomorphe, sont d'une grande utilité pour les besoins actuels
des données hébergées dans le \textit{Cloud}.\\

Nous nous intéressons dans ce mémoire à la sécurité de ces schémas SSE. Nous
commencerons par présenter leur fonctionnement puis étudierons leur sécurité en
définissant différents modèles. En nous appuyant sur l'article de Cash
\textit{et al.} \cite{ref2}, nous définirons un ensemble de profils de fuite
permettant de classifier les schémas SSE selon la quantité d'information qu'ils
révèlent. Nous présenterons par la suite différentes attaques réalisées en
fonction de ces profils de fuite et en donnerons les résultats.
