\chapter{Schémas de recherche sur données chiffrées}

\begin{quote}
{\small Apparus dans les années 2000 avec les premières applications liées au
\textit{Cloud}, les schémas de recherche sur données chiffrées, ou plus
simplement SSE pour \textit{Searchable Symmetric Encryption}, permettent à un
client de stocker et de rechercher de façon confidentielle et sécurisée ses
données en externe. Nous présentons dans ce premier chapitre le fonctionnement
de ces schémas.}
\end{quote}

\section{Notations}

Nous noterons $\Pr$ la probabilité d'un événement, $\vert v \vert$ la longueur
d'un vecteur $v$ et $\#E$ le cardinal d'un ensemble $E$. Nous rappelons qu'une
fonction $\mu : \mathbb{N} \rightarrow \mathbb{R}$ est dite \textit{négligeable}
en $\lambda$ si pour tout polynôme $p$ il existe $k \in \mathbb{N}$ tel
que $\mu(\lambda) \leqslant 1/\vert p(\lambda) \vert$ pour $\lambda > k$.

\subsection{Mots-clefs et documents}
Soit $\mathsf{W}=\{w_1,\dots,w_m\}$ un ensemble composé de $m$ mots-clefs. Nous
noterons $\mathsf{D}=\{d_1,\dots,d_n\}$ un ensemble de $n$ documents où chaque
document $d_i$ sera considéré comme une liste ordonnée de $\ell_i$ mots-clefs
appartenant à $\mathsf{W}$, c'est-à-dire $d_i=(w_1,\dots,w_{\ell_i}}) \in
\mathsf{W}^{\ell_i}$. Nous définissons, pour un ensemble documentaire
$\mathsf{D}$, la valeur $N$ correspondant au nombre total de couples où un
couple est composé d'un document et d'un mot-clef qu'il contient. En d'autres
termes, $N=\# \{(d,w):d \in \mathsf{D},w \in d\}$. Nous associons également à
chaque document $d_i$ l'ensemble noté $\mathsf{W}_i$ correspondant à l'ensemble
des mots-clefs distincts qui le composent.\\

Nous représentons par $\mathsf{D}^*=\{d_1^*,\dots,d_n^*\}$, l'ensemble des
documents chiffrés où $d_i^*$ est le chiffré du document $d_i$. De la même
manière, nous notons par $\mathsf{W}^*=\{w_1^*,\dots,w_m^*\}$, l'ensemble des
mots-clefs de $\mathsf{W}$ transformés par une fonction à sens-unique où $w_i^*$
est le transformé du mot-clef $w_i$.\\

Enfin, un document chiffré $d_i^*$ sera représenté de façon unique par son
identifiant $i$, il peut s'agir par exemple de son emplacement mémoire.

\begin{remark}
Pour simplifier les idées, nous considérons les documents comme des
``documents textes''. Il est toutefois important de garder à l'esprit que tout
ce qui suit fonctionne avec des documents arbitraires (e.g. des fichiers images
ou vidéos) tant qu'ils sont indexables.
\end{remark}

\subsection{Recherches sur données chiffrées et historique}
Soit $\mathsf{D}$ un ensemble de documents définis sur l'ensemble de mots-clefs
$\mathsf{W}$ et $w \in \mathsf{W}$. Une recherche par le mot-clef $w$ sur les
documents chiffrés (les éléments de $\mathsf{D}^*$) consiste à retrouver ceux
dont les clairs contiennent $w$.

\begin{defi}[Recherche sur un mot-clef]
Soit $\mathsf{W}$ un ensemble de mots-clefs et $w \in \mathsf{W}$, $\mathsf{D}$
un ensemble de documents définis sur $\mathsf{W}$. Nous définissons
$\mathsf{D}^*(w)$ comme l'ensemble des identifiants des documents chiffrés de
$\mathsf{D}^*$ dont les clairs contiennent le mot-clef $w$. C'est-à-dire
\begin{align*}
\mathsf{D}^*(w)=\{i:w \in \mathsf{W}_i\}
\end{align*}
\end{defi}

Plus généralement, nous pouvons définir une recherche booléenne.

\begin{defi}[Recherche booléenne]
Soit $\mathsf{D}$ un ensemble de documents définis sur l'ensemble de mots-clefs
$\mathsf{W}$. Soit $\bar{w} \subseteq \mathsf{W}$ un sous-ensemble de mots-clefs
et $\phi$ une formule booléenne sur $\bar{w}$. Nous définissons $\mathsf{D}^*
(\phi(\bar{w}))$ comme l'ensemble des identifiants des documents chiffrés dont
les clairs vérifient la formule booléenne $\phi(\bar{w})$. C'est-à-dire
\begin{align*}
\mathsf{D}^*(\phi(\bar{w}))=\{i:\phi(\bar{w})=\mathsf{True}\mathrm{\ pour\ }
\mathsf{W}_i\}
\end{align*}
\end{defi}

\begin{exmp}
Soit $\mathsf{D}=\{d_1,d_2\}$ un ensemble de deux documents définis sur
$\mathsf{W}=\{w_1,w_2,w_3\}$, où $d_1=(w_1,w_2,w_3,w_1)$ et $d_2=(w_1,w_2,w_2)$.
La recherche booléenne sur $\bar{w}=(w_1,w_3)$ définie par $\phi(\cdot)=(\cdot
\land \cdot)$ nous donne alors $\mathsf{D}^*(\phi(\bar{w}))=1$.
\end{exmp}

\begin{remark}
Il est à noter que certains schémas $\mathrm{SSE}$ n'autorisent que les
recherches par mot-clef unique. Dans ce cas, nous aurons simplement $\bar{w}=
\{w\}$ et $\phi(\bar{w})=w$. Afin de simplifier les notations et sauf mention du
contraire, nous parlerons dans la suite de ce mémoire de recherche par mot-clef
unique. Il est néanmoins important de se souvenir que certains schémas
$\mathrm{SSE}$ autorisent des recherches booléennes tel le schéma de Cash
\textit{et al.} présenté dans $\mathrm{\cite{ref9}}$.
\end{remark}

Nous définissons maintenant un historique comme la donnée d'un ensemble de
documents et d'une liste de recherches appliquées à cet ensemble documentaire.

\begin{defi}[Historique]
Soit $\mathsf{D}=\{d_1,\dots,d_n\}$ un ensemble de $n$ documents définis sur
$\mathsf{W}$. Un historique de $q$ recherches est un couple $H=(\mathsf{D},Q)$
où $Q=(w_1,\dots,w_q)$ avec $w_i \in \mathsf{W}$ pour $1 \leqslant i \leqslant
q$.
\end{defi}

\begin{exmp}
Soit $\mathsf{D}=\{d_1,d_2\}$ un ensemble de deux documents définis sur
$\mathsf{W}=\{w_1,w_2,w_3\}$, où $d_1=(w_2,w_1,w_1)$ et $d_2=(w_3,w_1)$. Un
historique de deux recherches peut être donc $H=(\mathsf{D},(w_1,w_3))$.
\end{exmp}

\subsection{Bases de données chiffrées}

Soit $\mathsf{D}=(d_1,\dots,d_n)$ un ensemble de $n$ documents définis sur
$\mathsf{W}$. Nous considérons la base de données chiffrée $\mathsf{EDB}$
associée à $\mathsf{D}$ comme une table liant l'ensemble des mots-clefs
transformés et les identifiants des documents chiffrés. Deux représentations
possibles pour une base de données sont présentées ci-dessous.\\

La première représentation repose sur un index. Nous faisons correspondre alors
à l'entrée $i$ de la table une liste de mots-clefs protégés (par exemple par une
fonction pseudo-aléatoire) référençant $d_i^*$.

\begin{exmp}
Soit deux documents $d_1=(w_1,w_3,w_2)$ et $d_2=(w_2)$. En utilisant la
représentation de la base de données chiffrée qui repose sur un index, nous
avons
\begin{center}
\begin{tabular}{l}
\multicolumn{1}{c}{$\mathsf{EDB}$} \\
\hline
$\mathsf{1} \leftarrow (w_1^*,w_3^*,w_2^*)$ \\
$\mathsf{2} \leftarrow (w_2^*)$
\end{tabular}
\end{center}
\end{exmp}

La seconde représentation repose sur un index inversé. Celle-ci, inversement à
la représentation via un index, associe un mot-clef transformé $w^*$ à un
ensemble d'identifiants de documents chiffrés dont les clairs contiennent le
mot-clef $w$, c'est-à-dire $\mathsf{D}^*(w)$.

\begin{exmp}
Soit deux documents $d_1=(w_1,w_3,w_2)$ et $d_2=(w_2)$. En utilisant la
représentation de la base de données chiffrée qui repose sur un index inversé,
nous avons
\begin{center}
\begin{tabular}{l}
\multicolumn{1}{c}{$\mathsf{EDB}$} \\
\hline
$w_1^* \rightarrow \{1\}$ \\
$w_2^* \rightarrow \{1,2\}$ \\
$w_3^* \rightarrow \{1\}$
\end{tabular}
\end{center}
\end{exmp}

\begin{remark}
Dans la suite, quand cela ne sera pas précisé, nous parlerons indifféremment
d'une base de données chiffrée reposant sur un index et d'une base de données
chiffrée reposant sur un index inversé.
\end{remark}

\section{Schémas de recherche statiques et dynamiques}

Nous rappelons qu'un schéma SSE fait intervenir un client et un serveur. Le
client possède un ensemble de documents sensibles qu'il souhaite chiffrer puis
exporter sur un serveur, de telle sorte que le serveur devra être capable
d'effectuer des recherches sur ces données chiffrées et retourner les
identifiants des documents chiffrés correspondant à la recherche initialement
donnée par le client. Un schéma SSE peut être statique ou bien dynamique. Un
schéma dynamique permet en plus d'initialiser une base de données chiffrée sur
le serveur, de la mettre à jour en ajoutant et/ou en enlevant certains documents
de manière consistante avec la fonction de recherche.

\subsection{Schémas de recherche statiques}
La première grande famille des schémas SSE est composée des schémas SSE dits
statiques. Ceux-là sont très utiles pour archiver un ensemble de documents qui
n'est pas appelé à être modifié par la suite.

\begin{defi}[Schéma statique]
Un schéma $\mathrm{SSE}$ statique $\mathsf{\Pi}$ de paramètre de sécurité
$\lambda$ est déterminé par un algorithme $\mathsf{\Pi}.\mathsf{Setup}$ et un
protocole $\mathsf{\Pi}.\mathsf{Search}$ où\\
\begin{quote}
\begin{itemize}
\item L'algorithme $\mathsf{\Pi}.\mathsf{Setup}$ prend en entrée le paramètre de
sécurité $\lambda$ et un ensemble de documents $\mathsf{D}$ définis sur
$\mathsf{W}$. Il génère une clef secrète $K_\lambda$ destinée au client
conduisant à la construction de la base de données chiffrée $\mathsf{EDB}$ et à
l'ensemble de documents chiffrés $\mathsf{D}^*$ à destination du serveur.
\begin{displaymath}
(K_\lambda,\mathsf{EDB},\mathsf{D}^*) \leftarrow \mathsf{\Pi}.\mathsf{Setup}
(\lambda,\mathsf{D})
\end{displaymath}
\item Le protocole $\mathsf{\Pi}.\mathsf{Search}$ prend en entrée la clef
secrète $K_\lambda$, un mot-clef $w \in \mathsf{W}$ et la base de données
chiffrée $\mathsf{EDB}$. La recherche par le mot-clef $w$ sera protégée à l'aide
de la clef $K_\lambda$ puis envoyée au serveur hébergeant la base de données
chiffrée. Le protocole se termine en retournant $\mathsf{D}^*(w)$ au client.
\begin{displaymath}
\mathsf{D}^*(w) \leftarrow \mathsf{\Pi}.\mathsf{Search}(K_\lambda,w,\mathsf{EDB})
\end{displaymath}
\end{itemize}
\end{quote}
\end{defi}

Intuitivement, nous disons qu'un schéma SSE statique $\mathsf{\Pi}$ est correct
si son protocole de recherche $\mathsf{\Pi}.\mathsf{Search}$ retourne l'ensemble
$\mathsf{D}^*(w)$ pour une recherche par le mot-clef $w$. Nous formalisons cette
notion par la définition ci-dessous.

\begin{defi}[Schéma de recherche statique correct]
Soit $\mathcal{A}$ un algorithme polynomial et probabiliste, et $\mathsf{\Pi}$
un schéma $\mathrm{SSE}$ statique de paramètre de sécurité $\lambda$. Nous
considérons l'expérience $\mathsf{CorSta}_\mathcal{A}^\mathsf{\Pi}(\lambda)$
présentée dans la figure \ref{corsta}.\\
Nous dirons que $\mathsf{\Pi}$ est correct si pour tout algorithme polynomial et
probabiliste $\mathcal{A}$
\begin{displaymath}
\Pr\left[\mathsf{CorSta}_\mathcal{A}^\mathsf{\Pi}(\lambda)=1\right] \leqslant
\mu(\lambda)
\end{displaymath}
où $\mu$ est une fonction négligeable en $\lambda$.
\end{defi}

\begin{figure}[h]
\centering
\begin{algorithm}[H]
$H \leftarrow \mathcal{A}(1^\lambda)$, $H=(\mathsf{D},Q)$ \;
$(K_\lambda,\mathsf{EDB},\mathsf{D}^*) \leftarrow \mathsf{\Pi}.\mathsf{Setup}
(\lambda,\mathsf{D})$ \;
\BlankLine
\For{$1 \leqslant i \leqslant q$}{
$\textsc{v}[i] \leftarrow \mathsf{\Pi}.\mathsf{Search}(K_\lambda,Q[i],
\mathsf{EDB})$ \;}
\BlankLine
\If{$\textsc{v}\not=(\mathsf{D}^*(Q[1]),\dots,\mathsf{D}^*(Q[q]))$}
{\textbf{retourner} $1$ \;}
\Else{\textbf{retourner} $0$\;}
\end{algorithm}
\caption{Expérience $\mathsf{CorSta}_\mathcal{A}^\mathsf{\Pi}(\lambda)$.}
\label{corsta}
\end{figure}

\subsection{Schémas de recherche dynamiques}
Contrairement aux schémas SSE statiques où le client ne peut qu'effectuer des
recherches sur des données chiffrées, les schémas SSE dynamiques permettent
également au serveur d'ajouter et de supprimer des documents à la base de
données chiffrée et ainsi de la mettre à jour de manière consistante avec le
protocole de recherche.

\begin{defi}[Schéma de recherche dynamique]
Un schéma $\mathrm{SSE}$ dynamique $\mathsf{\Pi}$ est défini de la même façon
qu'un schéma $\mathrm{SSE}$ statique mais possède en plus un protocole
$\mathsf{\Pi}.\mathsf{Update}$ où
\begin{quote}
Le protocole $\mathsf{\Pi}.\mathsf{Update}$ prend en entrée la base de données
chiffrée $\mathsf{EDB}$, la clef secrète du client $K_\lambda$, une opération de
mise à jour $\mathsf{op} \in \{\mathsf{add},\mathsf{del},\mathsf{edit^+},
\mathsf{edit^-}\}$, un document $d$ ainsi qu'un ensemble de mots-clefs
$\mathsf{W}_0$. L'opération de mise à jour, le document $d$ et l'ensemble de
mots-clefs $\mathsf{W}_0$ seront protégés à l'aide de la clef secrète du client
et envoyés au serveur. Le protocole met alors à jour la base de données chiffrée
et retourne $\mathsf{EDB'}$ au serveur.
\begin{align*}
\mathsf{EDB'} \leftarrow
\mathsf{Update}(\mathsf{EDB},K_\lambda,\mathsf{op},d,\mathsf{W}_0)
\end{align*}
\end{quote}
\end{defi}

Précisons l'effet des opérations.
\begin{itemize}
\item L'opération $\mathsf{add}$ associée au document $d$ et à un ensemble de
mots-clefs $\mathsf{W}_0$ permet au serveur d'ajouter $d^*$ à $\mathsf{D}^*$ et
de le référencer aux mots-clefs protégés de $\mathsf{W}_0$ dans la base de
données chiffrée.
\item De la même manière, l'opération $\mathsf{del}$ permet au serveur de
supprimer $d^*$ mais également de ne plus le référencer aux mots-clefs protégés
$\mathsf{W}_0^*$ dans la base de données chiffrée.
\item Les opérations $\mathsf{edit^+}$ et $\mathsf{edit^-}$ permettent au
serveur de modifier le référencement de $d^*$ en lui associant les mots-clefs
protégés de $\mathsf{W}_0$ via l'opération $\mathsf{edit^+}$ ou en les enlevant
avec l'opération $\mathsf{edit^-}$.
\end{itemize}

\begin{remark}
Si la définition d'un schéma dynamique correct est conceptuellement simple, sa
formalisation est plus délicate puisqu'il faut s'assurer que dans le jeu de
sécurité que tout se passe bien quelque soit l'enchaînement des recherches et
des opération de mise à jour.
\end{remark}
